\documentclass[]{article}
\usepackage{lmodern}
\usepackage{amssymb,amsmath}
\usepackage{ifxetex,ifluatex}
\usepackage{fixltx2e} % provides \textsubscript
\ifnum 0\ifxetex 1\fi\ifluatex 1\fi=0 % if pdftex
  \usepackage[T1]{fontenc}
  \usepackage[utf8]{inputenc}
\else % if luatex or xelatex
  \ifxetex
    \usepackage{mathspec}
    \usepackage{xltxtra,xunicode}
  \else
    \usepackage{fontspec}
  \fi
  \defaultfontfeatures{Mapping=tex-text,Scale=MatchLowercase}
  \newcommand{\euro}{€}
\fi
% use upquote if available, for straight quotes in verbatim environments
\IfFileExists{upquote.sty}{\usepackage{upquote}}{}
% use microtype if available
\IfFileExists{microtype.sty}{\usepackage{microtype}}{}
\usepackage{longtable,booktabs}
\ifxetex
  \usepackage[setpagesize=false, % page size defined by xetex
              unicode=false, % unicode breaks when used with xetex
              xetex]{hyperref}
\else
  \usepackage[unicode=true]{hyperref}
\fi
\hypersetup{breaklinks=true,
            bookmarks=true,
            pdfauthor={Team Blue Mondays},
            pdftitle={Failed case - HealthCare.gov},
            colorlinks=true,
            citecolor=blue,
            urlcolor=blue,
            linkcolor=magenta,
            pdfborder={0 0 0}}
\urlstyle{same}  % don't use monospace font for urls
\setlength{\parindent}{0pt}
\setlength{\parskip}{6pt plus 2pt minus 1pt}
\setlength{\emergencystretch}{3em}  % prevent overfull lines
\setcounter{secnumdepth}{0}

\title{Failed case - HealthCare.gov}
\author{Team Blue Mondays}
\date{Alex van Manen (Lead), Pim Tegelaar (Vice Lead), Job Witteman, Erik
Verhoofstad, Alberto Martinez de Murga, Ger Krijnen, Carla Alvarado \&
Joel James Bartholomew \hfill \break \break \centering{\today}}

\begin{document}
\maketitle

{
\hypersetup{linkcolor=black}
\setcounter{tocdepth}{3}
\tableofcontents
}
\newpage

\subsection{Abstract}\label{abstract}

On 23 March 2010, president Obama signed the Patient Protection and
Affordable Care Act, better known as Obamacare. One of the consequences
of this law was that everyone in the USA would require access to a
health insurance marketplace. Therefore the federal government started
the HealtCare.gov project. The launch of the website was on 1 October
2013, but from the beginning it has been plagued by technical
difficulties. The \$630 million website crashed a couple of minutes
after the launch. From the 9.47 million people that tried to sign up in
the first few weeks only 271,000 succeeded in doing so.

This report focuses on why this big project (partially) failed and which
software and project methodology principles were not applied. This is
done by first explaining the project's complexity, its organization and
identifying the reasons for the failure. After that, this report
analyses the reasons to determine the root causes and explains which
principles from RUP, Scrum and the Mars Lander project were applied and
which not.

\section{Initial complexity}\label{initial-complexity}

The construction of HealthCare.gov was overseen by Centers for Medicare
and Medicaid Services (CMS) which is a part of the US Department of
Health and Human Services (HHS). The primary contractor was CGI who had
a contract of around 250 million dollars. Next to that there were 16
official subcontractors, but the total number of subcontractors was
actually 55. All under the supervision of CMS. There were also 300
private insurers. All in all there were around 4000 plans.

The clients were according to {[}4, page 6{]}: the Department of Health
and Human Services (HHS), Centers for Medicare \& Medicaid Services
(CMS), 36 States, 300 private insurers, U.S. Chief Technology Office,
GAO, Media, Citizens, Social Security Administration, the Internal
Revenue Service, Veterans Administration, Office of Personnel
Management, Peace Corps, etc.

The internal architecture of HealthCare.gov is very complex. US
residents who want to apply for an insurance needed a lot of information
from the different systems. Like their income and immigration status.
The website needed to be connected with all the different systems from
the federal government and the private insurers, like the Internal
Revenue Service, Social Security Administration, the Peace Corps, etc.
The idea was the information of all the different federal systems should
be retrieved real time. According to {[}4{]} the U.S. Chief Technology
Officer Todd Park has said that the government expected HealthCare.gov
to draw 50,000 to 60,000 simultaneous users. But in the first week the
site was overwhelmed by up to five times as many users.

\section{Timeline of problems and
solutions}\label{timeline-of-problems-and-solutions}

\subsection{Before the launch}\label{before-the-launch}

There were a lot of warnings before the launch of the website by
different parties. But is seems nothing was done with them. Below are a
couple of them noted {[}4{]}.

On April 4, 2013 a 15-page document by McKinsey \& Co. warns among
others the CMS administrator that there was insufficient end-to-end
testing and advises that a limited initial launch of the website would
be ideal. It seems nothing is done with this report, because it surfaced
only half November 2013. One and a half month after the launch. Next to
that there were only two weeks of end-to-end testing before the launch
took place for all the users.

The Government Accountability Office (GAO), an independent organisation
of the government, reported in June 2013 that there could be trouble
with the goal of launching the website on the first of October 2013.

On 10 September, there was a panel of the House of Representatives who
were hearing among others consulting firm Leavitt Partners and
contractor CGI. CGI stated during this hearing that it was on schedule
for launching the website. On the other hand Leavitt Partners stated
that the launch will be `rocky' {[}4, page 7{]} because there are still
ongoing technology challenges for the online health insurance exchange.

According to the New York Times {[}4, page 7{]}, documents released by
House investigators reveal that the `testing bulletin' showed that the
website as of September 30 could handle only about 1100 users at a time,
even though officials have said it should have been able to accommodate
perhaps as many as 60,000 users.

\subsection{After the launch}\label{after-the-launch}

At its launch (1 October 2013) the website did not perform well. Only
271,000 of the 9.47 million users that tried to register were
successful. In 2014 the total costs of the project was 1.7 billion
dollar (reference). Suddenly, one question arises. What would have
happened if they had established a schedule of access to the site?.
There were official warnings saying that the site was going to be unable
to accommodate so many users at once.

An extra team is recruited (22th October 2013). It was recognized that
product testing was insufficient. Serious underestimation of time.

Quality Software Services, Inc. (QSSI) is selected as the contractor
responsible to now oversee federal website fixes (October 27, 2013). The
\emph{failure reason} points to a technical complexity (a single data
centre brings down the most critical mission, `the website' this should
be presumed and avoided). The \emph{failure factor} leans at
requirements that should be identified and addressed before project
completion.

(October 31, 2013) Again an unrealistic schedule. The site crashed for a
second time due to having a small amount of servers, an explicit gap at
project definition and planning.

(November 14, 2013) President Obama declares to the press that he wasn't
well informed of the situation. The project team was not prepared for
the failure. signed up for plans using the site compared to the 227,000
that had enrolled through the 14 state run exchanges.

Obama Administration set deadline of a working site for the `vast
majority' of users. (December 1, 2013)

\section{Failure causes \& possible
solutions}\label{failure-causes-possible-solutions}

The failure was caused by a complex interplay of problems:

\subsection{Cause 1: Contractor
Management}\label{cause-1-contractor-management}

\begin{enumerate}
\def\labelenumi{\arabic{enumi}.}
\itemsep1pt\parskip0pt\parsep0pt
\item
  Contractor delays and performance issues were not always identified.
\item
  A contractor incurred unauthorized costs that increased the cost of
  the contract.
\item
  Contracting officers in all government agencies did not have access to
  contractor past-performance evaluations when making contract awards.
\item
  Critical deliverables and management decisions were not properly
  documented.
\end{enumerate}

\subsection{Cause 2: Architecture, non-functional
requirements}\label{cause-2-architecture-non-functional-requirements}

\begin{enumerate}
\def\labelenumi{\arabic{enumi}.}
\itemsep1pt\parskip0pt\parsep0pt
\item
  The system was not designed to handle the massive influx of initial
  users.
\item
  Insufficient effort made to build a system that would meet the
  performance and availability needs of it's stakeholders.
\item
  Security solutions were seemingly slapped together in a shoddy manner
  instead of through the kind of systematic approach that is expected in
  a high quality software solution.
\item
  Many thousands of pages of legal healthcare regulations did not
  translate seamlessly into functional requirements.
\item
  The basic architecture was designed and built around the notion that
  the system would forward requests for quotes from insurance seekers to
  external vendors in real-time; however, this massive
  inter-connectivity and the subsequent burden on the government servers
  caused the system to collapse.
\end{enumerate}

\begin{quote}
However, all these points, extracted from {[}1{]} are a compilation of
the problem's triggers , but \textbf{where could it change?}
\end{quote}

Besides all the disastrous organizational perspective, and scoping at
the missing functional requirements, the elicitation, documentation and
communication of those requirements probably could have helped this
project. There are many structured techniques specifically designed to
learn others about the system and what should it carry, some techniques
such as `planguage'{[}10{]} or `volere templates'.

\subsection{Cause 3: Change Management}\label{cause-3-change-management}

The inability to handle a growing amount of work, its potential to be
enlarged in order to accommodate that growth, or a limited space to
accommodate iterations can be a very dangerous thing. In this case, just
a week before the deadline a requirement changed that had a major impact
on the architectural constraints of the site.

\subsection{Cause 4: Politics (Ostrich
effect)}\label{cause-4-politics-ostrich-effect}

Politicians were setting a deadline that was not realistic. Only two
weeks for the launch there were end-to-end tests. The requirement of
60,000 users was not met according to the New York Times on 30 September
2013 that based there statements on documents released by House
investigators {[}4{]}. Only 11,000 users could simultaneously access the
website. Also a McKinsey \& Co document, made on April 4 2013, that was
addressed among others to the CMS Administrator {[}4, page 7{]}. The
document stated there was insufficient time for end-to-end testing and
that a `limited initial launch' would be ideal. This report only
surfaced on 13 November 2013. One and a half month after the release of
the website. It is unclear why this information was not used to postpone
the deadline or to do a launch for a limited number of users.

\section{Contractor Management}\label{contractor-management}

\subsection{Key factors in failure}\label{key-factors-in-failure}

One of the most important reasons for the failure of the
\emph{HealthCare.gov} project, from the process and management point of
view, was a mismanagement of contractors.

The \emph{Center for Medicare \& Medicaid Services} (CMS) depended
heavily on contractors to develop and manage the \emph{Federal
marketplace}. Federal law prescribes precise rules that contractor
officers need to follow in order to manage the performance of the
contractors. The following mistakes were made during the project
{[}3{]}:

\begin{itemize}
\item
  \textbf{Contracting officers did not receive all contract deliverables
  and periodically neglected to use those to monitor the contractors
  performance.} Law enforces the contractor officers and their
  representatives to monitor the performance of their contractors.
  Sometimes reports were incomplete and deliverables handed in months
  after their deadlines, without any explanation requested by or given
  to the CMS.
\item
  \textbf{Unauthorised CMS personnel added work and increased the cost
  of one of the contracts.} Contractor officers added extra work to one
  contract without having the authority to issue such work, around 40
  work units were added for a total extra cost of \$28 million.
\item
  \textbf{Contract officer's representatives were not properly
  designated and authorised in written contracts.} 75\% of the contracts
  reviewed missed important information like the contracting officer's
  representative or the specific duties and responsibilities assigned
  for each contract.
\item
  \textbf{CMS's contracting officer representatives did not have the
  required certification.} Contracts that are valued at more than 10
  million dollar require a Level III certification in risk management.
  Not all contracting officer's representatives, who worked with
  contracts of this size possessed the required certification.
\item
  \textbf{CMS has not complied with the standards of ethical conduct in
  at least one occasion.} In one case, one of the CMS panel members
  responsible for awarding a contract had recently worked together with
  the party that applied for the contract, thus creating a possible
  conflict of interests. The employee did not raise this as an issue to
  his supervisor(s) and as such acted in conflict with the CMS's ethical
  code.
\item
  \textbf{Contracting officers did not always prepare contractor
  past-performance evaluations}. Some contracts were not registered at
  the institution that keeps a record of government contractors.
  Subsequently, the CMS did not perform performance reviews for those
  contracts that are required for the logs of that institution.
\end{itemize}

\subsection{Scrum}\label{scrum}

\subsubsection{Principles applied during the
project}\label{principles-applied-during-the-project}

A key part of the Scrum methodology is to prioritise communication over
documentation. Due to the all the legal requirements and licensing
required for big government project like this one scrum was not used as
process for the overall project. Some of the contractors, like
Development Seed {[}24{]}, used Agile methods for their development
internally. Unfortunately, the main problem for the contractors was in
their communication with the CMS, who were not involved on such a low
level of the project.

\subsubsection{Which principles could help
HealthCare.gov?}\label{which-principles-could-help-healthcare.gov}

According to the Scrum Guide
(http://www.scrumguides.org/scrum-guide.html\#team) Scrum is most
effective in small teams with up to 9 people. Considering the sheer size
of this project, a pure Scrum approach for the entire project would have
likely done more harm than good. However, some parts of Scrum could have
benefited the CMS.

Take for example a \textbf{product backlog} together with having a
dedicated \textbf{product owner}; These could have stopped unauthorized
CMS employees from changing the scope of the project. Any product that
needs to be added to backlog would have to be added by the product
owner, who can check if the request is legitimate ( assuming that all
product owners are authorized to change the scope of the contract ).

The main problem in this case was the lack of control on the contract
deliverables. Perhaps if the CMS worked with a Scrum flow which included
a \textbf{demo} phase, those deliverables could be incrementally shown
to the CMS stakeholders, even if they aren't software but documents
e.g.~security audit reports and progress reports. However, the report by
Daniel Levinson {[}3{]} makes it sound like the main cause for missing
these deliverables is because CMS was not enforcing them too strictly.
So for this approach to yield any success, the contract officers would
also need to take a more professional stance on the deliverables.

\subsection{Rational Unified Process
(RUP)}\label{rational-unified-process-rup}

\subsubsection{Principles applied during the
project}\label{principles-applied-during-the-project-1}

\begin{itemize}
\item
  Vision: The vision of what to build was clear, thousands of pages of
  legal documents provided documentation on what to build
  exactly.{[}1{]}
\item
  Plan: The product was planned to launch on 1 October 2013 and did so.
  Testing was planned for two weeks before launch {[}2{]}. The plan
  might not have been ideal, but it was there.
\item
  Risks: Laws on contracting dictate what has to be done to supervisor
  contractors to mitigate risks, however those laws were not always
  followed {[}3{]}. This means that this principle was not followed
  correctly.
\item
  Issues: Due to a lack of data, it is hard to tell what has been done
  in order to track issues.
\item
  Architecture: While there was an architecture set to integrate all the
  services, it was not designed to handle to load that it got on the
  launch day.{[}4{]}
\item
  Product: This step focuses mainly on building the product step by step
  and testing it during every step. In some cases this was done, with
  the front-end for example. The front-end was made by
  \href{https://developmentseed.org/}{Development Seed} and the
  non-interactive part of this went live before the rest of
  HealthCare.gov. This part of the HealthCare.gov website was generally
  well received and appreciated for its modern take on government
  websites {[}24{]}. While the system as a whole was only tested two
  weeks before launch. Which was too short to find all important bugs,
  and too close to launch to fix all the problems that were found.
\item
  Evaluation: This principle assumes an iterative approach, which was
  not present during the building of HealthCare.gov. So this principle
  was not followed at all.
\item
  Change Requests: The Obama administration kept modifying regulations
  and policies of the Affordable Care Act, which inherently meant that
  the scope of the HealthCare.gov website kept changing too. Which
  probably has caused some hardship with the changes trickling down to
  all the contractors, but there is a lack of documentation on how the
  changes were documented and implemented.
\end{itemize}

\subsubsection{Which principles could help
HealthCare.gov?}\label{which-principles-could-help-healthcare.gov-1}

As stated above, a stricter monitoring of the deliverables by the
contractors could have helped this project. What also could have
prevented the failure was a better focus on product and evaluation. If a
more Agile way of working was promoted, with small deliverables, that
can be tested. Then perhaps bugs and the wrong scale for the
architecture could have been identified earlier, while there was still
time to fix the problems. Instead of when the integration happened, two
weeks before the launch.

\subsection{Mars Lander}\label{mars-lander}

The Mars Lander's development approach focuses on durability and
reliability to the extreme. Judging by how the HealthCare.gov website
only worked correctly for 1\% of the visitors during their launch week,
this was not the case for the HealthCare.gov website.

Every part that is aboard the Mars Lander needs to go through a
qualification testing procedure of 12 to 15 months. And then when all
the components are combined there is another qualification testing
procedure which can take a few years {[}5{]}. This shows that the Mars
Lander team takes their integration tests very seriously, which was not
the case for the CMS with all the missing deliverables.

Of course there is a downside to this as well. If the CMS was to
implement qualification testing on this scale, costs for the project
would likely sky-rocket and the project would take longer. Then again,
the HealthCare.gov is not flying through space, where maintenance is
impossible to perform. So the CMS could definitely profit by looking how
NASA does its integration tests and implementing those, albeit on a
smaller scale.

\section{Architecture, non-functional
requirements}\label{architecture-non-functional-requirements}

\subsection{Key factors in failure}\label{key-factors-in-failure-1}

One of the primary problems with HealthCare.gov was its performance. The
architecture was designed to pass on requests from the web servers
real-time to the back-end systems. The system was not designed to be
able to handle the high loads that it faced after going to production
{[}13{]}.

Requirements that had a high impact on the architecture were delivered
late. One such requirement, was that users needed to log in, before they
could use the site: ``{[}customer representatives were{]} debating
whether consumers should be required to register and create
password-protected accounts before they could shop for health plans''.
This was delivered three months before the planned release date. There
was scant time, to incorporate these requirements {[}9{]}.

Another one of the problems the site faced, was that there were a number
of security vulnerabilities {[}18{]}. These came to light during a
security breach in the July of 2014. The breach was discovered by a CMS
security team in the following month. It occurred due to a development
server not being properly configured and this led to malware being
uploaded to the system.

\subsection{Software development
methodologies}\label{software-development-methodologies}

Could the problems stated above have been fixed by using software
methodologies?

\subsection{Scrum}\label{scrum-1}

The back-end of HealthCare.gov is reported to have been built using
agile practices but they were likely not applied correctly due to the
project having explicit phases defined such as testing {[}12{]}. In
Scrum, architecture is designed as required for the next production
increment {[}16{]}. This does not necessarily lead to the architecture
becoming more flexible. If a Scrum approach would have been implemented
properly, a minimal viable product would have been created first, which
should have already contained the performance requirements. The
different teams working on the project should have agreed on a
definition of done. A large amount of work on HealthCare.gov was left
unfinished during the launch {[}20{]}. A clear and formal definition of
done would have resulted in an architecture containing all quality
attributes of the system. However, a minimal viable product might not
have included integration with all systems, which is where the
performance really starts to hurt.

On the other side of the spectrum, the front-end of HealthCare.gov is
considered to be a success. A start-up known as Development Seed was
responsible for the front-end of the website and made use of agile
practices {[}12{]}.

\section{RUP}\label{rup}

Contractors stated ``\ldots{}mere days before the launch date, {[}CMS{]}
tested the system's ability to handle tens of thousands of users at
once. It crashed after a few hundred.'' {[}9{]}. This likely indicates
an issue with validating the architecture.

Next to iterative development, RUP also defines phases of a project's
life-cycle {[}15{]}. In the elaboration phase, most of the architecture
should be done. At HealthCare.gov most of the architecture was also done
up front. This did not help them foresee the problems that occurred
during the go-live. RUP does promote a risk first approach. If RUP would
have been properly implemented, the riskier parts of the process, such
as integration and performance, would have been tackled in the
beginning.

\subsection{Mars Lander}\label{mars-lander-1}

The software methodology used to build the Mars rover is based on
risk-reduction principles {[}19{]}. The U.S. government has built a
large number of software systems and thus has a history of security
vulnerabilities that can possibly occur. One of the security
vulnerabilities found was due to a development system being configured
with default credentials which is a mistake made by the developers. This
made it possible for an automated attacker to take advantage of the
system and thus use it to upload malware {[}18{]}. Apparently this
weak-link in the security was not identified in the architecture of
HealthCare.gov. If CMS were to have taken measures to reduce the amount
risk involved when building the system, this would have have likely
prevented the vulnerability, simply due to it being an extremely common
error. The development system would have been properly identified as a
point of weakness. Because of this, it would not have been insecurely
configured.

\section{Change Management}\label{change-management}

In order to shop for a healthcare plan, consumers had to register on the
site for an account first. It was reported that the response times of
the registration pages were very high with load times up to 71 seconds
{[}9,21{]}. This symptom occurred because architecture was not able to
handle the concurrent users for the registration pages. Around September
2013, just weeks before the deadline, a decision was made that all users
had to register first, before they could start shopping for a health
plan {[}22{]}. Because of this change the registration pages became a
bottleneck for the entire site. The changing functional requirement had
a major impact on the architecture of the site.

\subsection{Key factor in failure}\label{key-factor-in-failure}

Many failures regarding performance were due to a poor architectural
design in the first place which in turn was caused by the omission of
well defined quality constraints {[}1{]}. This was discussed in the
previous chapter. However, the cause for the performance problems at the
registration pages was a requirement that changed in a late stage during
the development. The requirement had a major impact on the architectural
constraints of the site that was not captured. Changing requirements
during the course of a project is a common phenomenon in software
engineering. We analysed whether Scrum or RUP have proper mechanisms to
deal with this and if these mechanisms could have prevented the problems
at HealthCare.gov.

\begin{longtable}[c]{@{}ll@{}}
\toprule\addlinespace
& Description
\\\addlinespace
\midrule\endhead
Symptom & High response times for the registration pages.
\\\addlinespace
Failure & The architecture could not handle the many concurrent users.
\\\addlinespace
Cause & A few weeks before the deadline a requirement changed that
stated that consumers
\\\addlinespace
& must register before able to shop for health plans.
\\\addlinespace
Context & Changing requirements with major architectural impact.
\\\addlinespace
Problem & No mechanism to deal with changing requirements.
\\\addlinespace
\bottomrule
\end{longtable}

\subsection{Scrum}\label{scrum-2}

In Scrum requirements are expressed in Product Backlog Items and are
maintained by a Product Owner who is also the sole person responsible
for them. These Product Backlog Items can change any time up until the
moment they become `in sprint' and are being developed by the
Development Team. It is during the Sprint Planning at the start of each
Sprint that the Development Team decides on how to implement the Product
Backlog Items requested by the Product Owner {[}17{]}. Scrum however
does not specify what the `how' should answer in terms of quality
constraints. Capturing the impact of the change at HealthCare.gov would
have been greatly dependent of the expertise of the Development Team or
Product Owner.

Another principle of Scrum is that at the end of each Sprint the team
delivers a fully tested and deployable product. In general, if this were
the case at HealthCare.gov the performance failure would still have
occurred but it would be detected within the duration of a sprint. The
project management would than have more time to handle the problem. Note
however, that this specific requirement change was made weeks before the
final deadline so having sprints would have made no difference either.

Overall Scrum does not provide well defined mechanisms to handle quality
assurance needed when dealing with changing requirements.

\subsection{RUP}\label{rup-1}

The Rational Unified Process (RUP) is an iterative software development
process in which risk management is an important feature. One of the ten
essentials `Change Requests' specifically deals with changing
requirements by providing a process for risk and impact analysis for any
proposed change {[}23{]}.

\begin{quote}
``The benefit of Change Requests is that they provide a record of
decisions, and, due to their assessment process, ensure that impacts of
the potential change are understood by all project team members. The
Change Requests are essential for managing the scope of the project, as
well as \textbf{assessing the impact of proposed changes.}'' {[}23{]}
\end{quote}

It is this part of RUP that could have prevented the performance
problems during registrations. Had the change been properly analysed on
its impact it could have on the architecture, the project team could
have decided to make the proper adjustments on the architecture or
reject the change at all.

\section{Sources}\label{sources}

{[}1{]} Cleland-Huang, Jane.
\href{http://ieeexplore.ieee.org/stamp/stamp.jsp?tp=\&arnumber=6774318}{Don't
Fire the Architect! Where Were the Requirements?} Software, IEEE 31.2
(2014): 27-29.

{[}2{]} Morgan, David \& Humer, Caroline. ``Timeline: U.S. healthcare
law's technology breakdown'' Reuters, 30 October 2013, Web. 9 February
2015.

{[}3{]} Levinson, Daniel. ``CMS Did Not Always Manage and Oversee
Contractor Performance for the Federal Marketplace as Required by
Federal Requirements and Contract Terms'', Office of Inspector General,
September 2015, Report.

{[}4{]} Anthopoulos, L., et
al.\href{http://www.sciencedirect.com/science/article/pii/S0740624X15000799}{Why
e-government projects fail? An analysis of the Healthcare.gov website},
Government Information Quarterly (2015).

{[}5{]} Harwoord, William. ``Slow, but rugged, Curiosity's computer was
built for Mars'' CNET, 10 August, 2012, Web. 12 February 2015.

{[}6{]} .,
\href{http://oig.hhs.gov/oas/reports/region3/31403001.pdf}{The Office of
Audit Services (OAS) provides auditing services for HHS}.

{[}7{]} .,
\href{http://www.computer.org/csdl/mags/ic/2014/06/mic2014060085.pdf}{The
Failure of HealthCare.gov Exposes Silicon Valley Secrets}.

{[}8{]} Government Accountability Office (GAO) (2013). ``Patient
protection and affordable care act: Status of CMS efforts to establish
federally facilitated health insurance exchanges.'' (GAO-13-601), 19
June 2013.

{[}9{]} Chambers \& Associates Pty Ltd,
\href{http://www.chambers.com.au/public_resources/case_study/obamacare/saving-obamacare-case-study-analysis.pdf}{Case
Study Analysis Saving Obamacare}.

{[}10{]} Tom Gilb (Planguage). (2001) Intel Corporation by Erik Simmons
\href{http://www.clearspecs.com/downloads/ClearSpecs20V01_Quantifying\%20Quality\%20Requirements.pdf}{``Quantifying
Quality Requirements Using Planguage''}.

{[}11{]} .,
\href{https://science.house.gov/legislation/hearings/full-committee-hearing-my-data-healthcaregov-secure}{Full
Committee Hearing - Is My Data on Healthcare.gov Secure?}, transcribed
\href{https://www.gpo.gov/fdsys/pkg/CHRG-113hhrg86893/pdf/CHRG-113hhrg86893.pdf}{here}.

{[}12{]} A. Jeffries,
\href{http://www.theverge.com/2013/10/8/4814098/why-did-the-tech-savvy-obama-administration-launch-a-busted-healthcare-website}{``Why
Obama's Healthcare.gov launch was doomed to fail'', The Verge, 2013}.

{[}13{]} USA TODAY,
\href{http://www.usatoday.com/story/news/nation/2013/10/05/health-care-website-repairs/2927597/}{``Obama
adviser: Demand overwhelmed HealthCare.gov'', 2016}.

{[}14{]} M. Heusser,
\href{http://www.cio.com/article/2380827/developer/developer-6-software-development-lessons-from-healthcare-gov-s-failed-launch.html}{``6
Software Development Lessons From Healthcare.gov's Failed Launch'', CIO,
2016}.

{[}15{]} P. Kruchten, The rational unified process. Reading, Mass:
Addison-Wesley, 1999.

{[}16{]} Schwaber, Ken. Scrum development process, 1997.

{[}17{]} Ken Schwaber, Jeff Sutherland,
\href{http://www.scrumguides.org/docs/scrumguide/v1/scrum-guide-us.pdf}{``The
Scrum Guide'', Scrum Alliance, 2013}.

{[}18{]} Cbsnews.com,
\href{http://www.cbsnews.com/news/critical-flaw-found-in-healthcare-gov-security/}{``Critical''
flaw found in HealthCare.gov security, 2016}.

{[}19{]} A. Scoica, ``Profile Benjamin Cichy'', XRDS: Crossroads, The
ACM Magazine for Students, vol.~20, no. 3, pp.~68-69, 2014.

{[}20{]} B. Ehley,
\href{http://www.thefiscaltimes.com/2015/03/05/After-2-Years-and-21-Billion-HealthCaregov-Unfinished}{``After
2 Years and \$2.1 Billion, HealthCare.gov Is Unfinished'', The Fiscal
Times, 2016}.

{[}21{]} AppDynamics, Jim Hirschauer,
\href{http://www.appdynamics.com/blog/apm/technical-deep-dive-whats-impacting-healthcare-gov/}{``Technical
deep dive on what's impacting Healthcare.gov'', 2013}.

{[}22{]} The New York Times,
\href{http://www.nytimes.com/2013/10/13/us/politics/from-the-start-signs-of-trouble-at-health-portal.html}{``From
the Start, Signs of Trouble at Health Portal'', 2013}.

{[}23{]} Leslee Probasco, ``The Ten Essentials of RUP - the Essence of
an Effective Development Process'', Rational Software White Paper, 2002.

{[}24{]} Ford, Paul. ``The Obamacare Website Didn't Have to Fail. How to
Do Better Next Time'', 16 October 2013, Web. 13 February 2015.

{[}25{]} Robertson. ``Mastering the Requirements Process'' Pearson
Education (Us) augustus 2012, Chapter 16, page 353.

\end{document}
